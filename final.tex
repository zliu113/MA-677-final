\documentclass[]{article}
\usepackage{lmodern}
\usepackage{amssymb,amsmath}
\usepackage{ifxetex,ifluatex}
\usepackage{fixltx2e} % provides \textsubscript
\ifnum 0\ifxetex 1\fi\ifluatex 1\fi=0 % if pdftex
  \usepackage[T1]{fontenc}
  \usepackage[utf8]{inputenc}
\else % if luatex or xelatex
  \ifxetex
    \usepackage{mathspec}
  \else
    \usepackage{fontspec}
  \fi
  \defaultfontfeatures{Ligatures=TeX,Scale=MatchLowercase}
\fi
% use upquote if available, for straight quotes in verbatim environments
\IfFileExists{upquote.sty}{\usepackage{upquote}}{}
% use microtype if available
\IfFileExists{microtype.sty}{%
\usepackage{microtype}
\UseMicrotypeSet[protrusion]{basicmath} % disable protrusion for tt fonts
}{}
\usepackage[margin=1in]{geometry}
\usepackage{hyperref}
\hypersetup{unicode=true,
            pdftitle={Final Project},
            pdfauthor={Zhaobin Liu},
            pdfborder={0 0 0},
            breaklinks=true}
\urlstyle{same}  % don't use monospace font for urls
\usepackage{color}
\usepackage{fancyvrb}
\newcommand{\VerbBar}{|}
\newcommand{\VERB}{\Verb[commandchars=\\\{\}]}
\DefineVerbatimEnvironment{Highlighting}{Verbatim}{commandchars=\\\{\}}
% Add ',fontsize=\small' for more characters per line
\usepackage{framed}
\definecolor{shadecolor}{RGB}{248,248,248}
\newenvironment{Shaded}{\begin{snugshade}}{\end{snugshade}}
\newcommand{\KeywordTok}[1]{\textcolor[rgb]{0.13,0.29,0.53}{\textbf{#1}}}
\newcommand{\DataTypeTok}[1]{\textcolor[rgb]{0.13,0.29,0.53}{#1}}
\newcommand{\DecValTok}[1]{\textcolor[rgb]{0.00,0.00,0.81}{#1}}
\newcommand{\BaseNTok}[1]{\textcolor[rgb]{0.00,0.00,0.81}{#1}}
\newcommand{\FloatTok}[1]{\textcolor[rgb]{0.00,0.00,0.81}{#1}}
\newcommand{\ConstantTok}[1]{\textcolor[rgb]{0.00,0.00,0.00}{#1}}
\newcommand{\CharTok}[1]{\textcolor[rgb]{0.31,0.60,0.02}{#1}}
\newcommand{\SpecialCharTok}[1]{\textcolor[rgb]{0.00,0.00,0.00}{#1}}
\newcommand{\StringTok}[1]{\textcolor[rgb]{0.31,0.60,0.02}{#1}}
\newcommand{\VerbatimStringTok}[1]{\textcolor[rgb]{0.31,0.60,0.02}{#1}}
\newcommand{\SpecialStringTok}[1]{\textcolor[rgb]{0.31,0.60,0.02}{#1}}
\newcommand{\ImportTok}[1]{#1}
\newcommand{\CommentTok}[1]{\textcolor[rgb]{0.56,0.35,0.01}{\textit{#1}}}
\newcommand{\DocumentationTok}[1]{\textcolor[rgb]{0.56,0.35,0.01}{\textbf{\textit{#1}}}}
\newcommand{\AnnotationTok}[1]{\textcolor[rgb]{0.56,0.35,0.01}{\textbf{\textit{#1}}}}
\newcommand{\CommentVarTok}[1]{\textcolor[rgb]{0.56,0.35,0.01}{\textbf{\textit{#1}}}}
\newcommand{\OtherTok}[1]{\textcolor[rgb]{0.56,0.35,0.01}{#1}}
\newcommand{\FunctionTok}[1]{\textcolor[rgb]{0.00,0.00,0.00}{#1}}
\newcommand{\VariableTok}[1]{\textcolor[rgb]{0.00,0.00,0.00}{#1}}
\newcommand{\ControlFlowTok}[1]{\textcolor[rgb]{0.13,0.29,0.53}{\textbf{#1}}}
\newcommand{\OperatorTok}[1]{\textcolor[rgb]{0.81,0.36,0.00}{\textbf{#1}}}
\newcommand{\BuiltInTok}[1]{#1}
\newcommand{\ExtensionTok}[1]{#1}
\newcommand{\PreprocessorTok}[1]{\textcolor[rgb]{0.56,0.35,0.01}{\textit{#1}}}
\newcommand{\AttributeTok}[1]{\textcolor[rgb]{0.77,0.63,0.00}{#1}}
\newcommand{\RegionMarkerTok}[1]{#1}
\newcommand{\InformationTok}[1]{\textcolor[rgb]{0.56,0.35,0.01}{\textbf{\textit{#1}}}}
\newcommand{\WarningTok}[1]{\textcolor[rgb]{0.56,0.35,0.01}{\textbf{\textit{#1}}}}
\newcommand{\AlertTok}[1]{\textcolor[rgb]{0.94,0.16,0.16}{#1}}
\newcommand{\ErrorTok}[1]{\textcolor[rgb]{0.64,0.00,0.00}{\textbf{#1}}}
\newcommand{\NormalTok}[1]{#1}
\usepackage{longtable,booktabs}
\usepackage{graphicx,grffile}
\makeatletter
\def\maxwidth{\ifdim\Gin@nat@width>\linewidth\linewidth\else\Gin@nat@width\fi}
\def\maxheight{\ifdim\Gin@nat@height>\textheight\textheight\else\Gin@nat@height\fi}
\makeatother
% Scale images if necessary, so that they will not overflow the page
% margins by default, and it is still possible to overwrite the defaults
% using explicit options in \includegraphics[width, height, ...]{}
\setkeys{Gin}{width=\maxwidth,height=\maxheight,keepaspectratio}
\IfFileExists{parskip.sty}{%
\usepackage{parskip}
}{% else
\setlength{\parindent}{0pt}
\setlength{\parskip}{6pt plus 2pt minus 1pt}
}
\setlength{\emergencystretch}{3em}  % prevent overfull lines
\providecommand{\tightlist}{%
  \setlength{\itemsep}{0pt}\setlength{\parskip}{0pt}}
\setcounter{secnumdepth}{0}
% Redefines (sub)paragraphs to behave more like sections
\ifx\paragraph\undefined\else
\let\oldparagraph\paragraph
\renewcommand{\paragraph}[1]{\oldparagraph{#1}\mbox{}}
\fi
\ifx\subparagraph\undefined\else
\let\oldsubparagraph\subparagraph
\renewcommand{\subparagraph}[1]{\oldsubparagraph{#1}\mbox{}}
\fi

%%% Use protect on footnotes to avoid problems with footnotes in titles
\let\rmarkdownfootnote\footnote%
\def\footnote{\protect\rmarkdownfootnote}

%%% Change title format to be more compact
\usepackage{titling}

% Create subtitle command for use in maketitle
\newcommand{\subtitle}[1]{
  \posttitle{
    \begin{center}\large#1\end{center}
    }
}

\setlength{\droptitle}{-2em}

  \title{Final Project}
    \pretitle{\vspace{\droptitle}\centering\huge}
  \posttitle{\par}
    \author{Zhaobin Liu}
    \preauthor{\centering\large\emph}
  \postauthor{\par}
      \predate{\centering\large\emph}
  \postdate{\par}
    \date{5/4/2019}


\begin{document}
\maketitle

\subsection{Statistics and the Law}\label{statistics-and-the-law}

H0: Rate of white applicants are the same as minority applicant H1: Rate
of white applicants are different

Assuming the alpha level is 0.05.We will conduct a two-sample t-test to
test the argument whether there is sufficient evidence for
discrimination. Then we will do another two-sample t-test to test
whether there is sufficient evidence for discriminatiion bewteen high
income white and high income minority.

\begin{Shaded}
\begin{Highlighting}[]
\NormalTok{acorn <-}\KeywordTok{read.csv}\NormalTok{(}\StringTok{"acorn.csv"}\NormalTok{)}
\KeywordTok{boxplot}\NormalTok{(acorn}\OperatorTok{$}\NormalTok{MIN,acorn}\OperatorTok{$}\NormalTok{WHITE)}
\end{Highlighting}
\end{Shaded}

\includegraphics{final_files/figure-latex/assumption-1.pdf}

\begin{Shaded}
\begin{Highlighting}[]
\NormalTok{ftest <-}\StringTok{ }\KeywordTok{var.test}\NormalTok{(acorn}\OperatorTok{$}\NormalTok{MIN,acorn}\OperatorTok{$}\NormalTok{WHITE)}
\NormalTok{ftest}
\end{Highlighting}
\end{Shaded}

\begin{verbatim}
## 
##  F test to compare two variances
## 
## data:  acorn$MIN and acorn$WHITE
## F = 2.8026, num df = 19, denom df = 19, p-value = 0.02993
## alternative hypothesis: true ratio of variances is not equal to 1
## 95 percent confidence interval:
##  1.109297 7.080589
## sample estimates:
## ratio of variances 
##           2.802583
\end{verbatim}

We can see that the p-value of F-test is p = 0.02993. It is less than
the significance level alpha which is 0.05. Thus, there is significant
difference between the variances of the two sets of data. We can use
t-test witch assume unequal variances.

\begin{Shaded}
\begin{Highlighting}[]
\NormalTok{res <-}\StringTok{ }\KeywordTok{t.test}\NormalTok{(acorn}\OperatorTok{$}\NormalTok{MIN, acorn}\OperatorTok{$}\NormalTok{WHITE, }\DataTypeTok{var.equal =} \OtherTok{FALSE}\NormalTok{,}\DataTypeTok{alternative =} \StringTok{"greater"}\NormalTok{)}
\NormalTok{res}
\end{Highlighting}
\end{Shaded}

\begin{verbatim}
## 
##  Welch Two Sample t-test
## 
## data:  acorn$MIN and acorn$WHITE
## t = 6.2533, df = 31.028, p-value = 2.979e-07
## alternative hypothesis: true difference in means is greater than 0
## 95 percent confidence interval:
##  15.49313      Inf
## sample estimates:
## mean of x mean of y 
##   36.8815   15.6250
\end{verbatim}

Conclusion:

The p-value of the test is 2.979e-07 that is less than the significance
level alpha which is 0.05. We reject the null hypothesis and support the
argument that the data are sufficient evidence of discrimination to
warrant corrective action.

\subsection{Comparing suppliers}\label{comparing-suppliers}

We will use chi-square test to conduct the analysis since the quality of
ornithopters are categorical, instead of continuous variable

Ho: ornithopters made by all three schools produce same qualities Ha:
ornithopters made by all three shcools produce different qualities

\begin{Shaded}
\begin{Highlighting}[]
\NormalTok{table <-}\StringTok{ }\KeywordTok{matrix}\NormalTok{(}\KeywordTok{c}\NormalTok{(}\DecValTok{12}\NormalTok{,}\DecValTok{23}\NormalTok{,}\DecValTok{89}\NormalTok{,}\DecValTok{8}\NormalTok{,}\DecValTok{12}\NormalTok{,}\DecValTok{62}\NormalTok{,}\DecValTok{21}\NormalTok{,}\DecValTok{30}\NormalTok{,}\DecValTok{119}\NormalTok{),}\DataTypeTok{ncol=}\DecValTok{3}\NormalTok{,}\DataTypeTok{nrow =} \DecValTok{3}\NormalTok{,}\DataTypeTok{byrow=}\OtherTok{TRUE}\NormalTok{)}
\KeywordTok{colnames}\NormalTok{(table) <-}\StringTok{ }\KeywordTok{c}\NormalTok{(}\StringTok{"Dead"}\NormalTok{,}\StringTok{"Art"}\NormalTok{,}\StringTok{"Fly"}\NormalTok{)}
\KeywordTok{rownames}\NormalTok{(table) <-}\StringTok{ }\KeywordTok{c}\NormalTok{(}\StringTok{"Area51"}\NormalTok{,}\StringTok{"BDV"}\NormalTok{,}\StringTok{"Giffen"}\NormalTok{)}
\NormalTok{table <-}\StringTok{ }\KeywordTok{as.table}\NormalTok{(table)}

\KeywordTok{chisq.test}\NormalTok{(table,}\DataTypeTok{correct =}\NormalTok{ F)}
\end{Highlighting}
\end{Shaded}

\begin{verbatim}
## 
##  Pearson's Chi-squared test
## 
## data:  table
## X-squared = 1.3006, df = 4, p-value = 0.8613
\end{verbatim}

Since the p-value is 0.8613 greater than 0.05, we fail to reject the
null hypothesis. We conclude that ornithopters made by all three schools
produce same qualities.

\subsection{How deadly are sharks}\label{how-deadly-are-sharks}

\begin{Shaded}
\begin{Highlighting}[]
\NormalTok{shark <-}\StringTok{ }\KeywordTok{read.csv}\NormalTok{(}\StringTok{"sharkattack.csv"}\NormalTok{)}
\NormalTok{p <-}\StringTok{ }\NormalTok{shark}\OperatorTok\KeywordTok{filter}\NormalTok{(Country.code}\OperatorTok{==}\StringTok{"US"}\OperatorTok{|}\NormalTok{Country.code}\OperatorTok{==}\StringTok{"AU"}\NormalTok{)}\OperatorTok
\StringTok{  }\KeywordTok{filter}\NormalTok{(Type}\OperatorTok{==}\StringTok{"Provoked"}\OperatorTok{|}\NormalTok{Type}\OperatorTok{==}\StringTok{"Unprovoked"}\NormalTok{)}
\NormalTok{p_table<-}\KeywordTok{table}\NormalTok{(}\KeywordTok{droplevels}\NormalTok{(p}\OperatorTok{$}\NormalTok{Type),}\KeywordTok{droplevels}\NormalTok{(p}\OperatorTok{$}\NormalTok{Country))}
\NormalTok{coul =}\StringTok{ }\KeywordTok{brewer.pal}\NormalTok{(}\DecValTok{3}\NormalTok{, }\StringTok{"Pastel2"}\NormalTok{) }
\NormalTok{data_percentage=}\KeywordTok{apply}\NormalTok{(p_table, }\DecValTok{2}\NormalTok{, }\ControlFlowTok{function}\NormalTok{(x)\{x}\OperatorTok{*}\DecValTok{100}\OperatorTok{/}\KeywordTok{sum}\NormalTok{(x,}\DataTypeTok{na.rm=}\NormalTok{T)\})}
\KeywordTok{barplot}\NormalTok{(data_percentage,}\DataTypeTok{xlab=}\StringTok{"group"}\NormalTok{, }\DataTypeTok{main =} \StringTok{"provoked v.s. unprovoked"}\NormalTok{)}
\end{Highlighting}
\end{Shaded}

\includegraphics{final_files/figure-latex/data exploration-1.pdf}

By looking at the comparsion between Australia and United States on
provked vs unprovoked, we can see that they are at similiar proportion.
Provking has about 15\% of all shark attacks in Australia and 12\% of
shark attacks in the United States.

\begin{Shaded}
\begin{Highlighting}[]
\NormalTok{fatal<-p<-shark}\OperatorTok\KeywordTok{filter}\NormalTok{(Country.code}\OperatorTok{==}\StringTok{"US"}\OperatorTok{|}\NormalTok{Country.code}\OperatorTok{==}\StringTok{"AU"}\NormalTok{)}\OperatorTok
\StringTok{  }\KeywordTok{filter}\NormalTok{(Fatal}\OperatorTok{==}\StringTok{"Y"}\OperatorTok{|}\NormalTok{Fatal}\OperatorTok{==}\StringTok{"N"}\NormalTok{)}
\NormalTok{fatal<-}\KeywordTok{table}\NormalTok{(}\KeywordTok{droplevels}\NormalTok{(fatal}\OperatorTok{$}\NormalTok{Country),}\KeywordTok{droplevels}\NormalTok{(fatal}\OperatorTok{$}\NormalTok{Fatal))}

\NormalTok{knitr}\OperatorTok{::}\KeywordTok{kable}\NormalTok{(fatal)}
\end{Highlighting}
\end{Shaded}

\begin{longtable}[]{@{}lrr@{}}
\toprule
& N & Y\tabularnewline
\midrule
\endhead
Australia & 879 & 318\tabularnewline
United States & 1795 & 217\tabularnewline
\bottomrule
\end{longtable}

By looking at the fatal versus non-fatal sharks attacks in these two
countrie, we can see that there are more sharks attacks in the United
States. On the other hand, the proportion of fatal attacks in Australia
(26.5\%) is much higher than that in the United States (10\%). To
further test whether Sharks Australia are more deadly or fatal than
those in the United States, we will conduct the chi-square test.

\begin{Shaded}
\begin{Highlighting}[]
\KeywordTok{chisq.test}\NormalTok{(fatal,}\DataTypeTok{correct =}\NormalTok{ F)}
\end{Highlighting}
\end{Shaded}

\begin{verbatim}
## 
##  Pearson's Chi-squared test
## 
## data:  fatal
## X-squared = 134.54, df = 1, p-value < 2.2e-16
\end{verbatim}

\begin{Shaded}
\begin{Highlighting}[]
\NormalTok{prob <-}\StringTok{ }\KeywordTok{matrix}\NormalTok{(}\KeywordTok{c}\NormalTok{(}\FloatTok{0.2739171}\NormalTok{,}\FloatTok{0.5593643}\NormalTok{,}\FloatTok{0.09909629}\NormalTok{,}\FloatTok{0.06762231}\NormalTok{), }\DataTypeTok{nrow=}\DecValTok{2}\NormalTok{, }
               \DataTypeTok{dimnames =} \KeywordTok{list}\NormalTok{(}\KeywordTok{c}\NormalTok{(}\StringTok{"Australi"}\NormalTok{,}\StringTok{"US"}\NormalTok{),}\KeywordTok{c}\NormalTok{(}\StringTok{"NonFatal"}\NormalTok{,}\StringTok{"Fatal"}\NormalTok{)))}
\NormalTok{prob}
\end{Highlighting}
\end{Shaded}

\begin{verbatim}
##           NonFatal      Fatal
## Australi 0.2739171 0.09909629
## US       0.5593643 0.06762231
\end{verbatim}

\begin{Shaded}
\begin{Highlighting}[]
\NormalTok{total<-}\DecValTok{879}\OperatorTok{+}\DecValTok{318}\OperatorTok{+}\DecValTok{1795}\OperatorTok{+}\DecValTok{217}
\KeywordTok{pwr.chisq.test}\NormalTok{(}\DataTypeTok{w =} \KeywordTok{ES.w2}\NormalTok{(prob), }\DataTypeTok{N =}\NormalTok{ total, }\DataTypeTok{df =} \DecValTok{1}\NormalTok{, }\DataTypeTok{sig.level =} \FloatTok{0.05}\NormalTok{)}
\end{Highlighting}
\end{Shaded}

\begin{verbatim}
## 
##      Chi squared power calculation 
## 
##               w = 0.2047583
##               N = 3209
##              df = 1
##       sig.level = 0.05
##           power = 1
## 
## NOTE: N is the number of observations
\end{verbatim}

From the chi-square test, we have sufficient evidence to say that that
sharks attacks in Australia is more fatal than that of in the United
States even though the number of attacks in the United States is higher
than that of Australia. With sample size equal to 3209, the statistical
power of the test chi-square test is 1.

\subsection{Power Analysis}\label{power-analysis}

The arcsine transformation is calculated by two times the arcsine of the
square root of the proportion.We can transform the proportional
parameter from (0,1) to (−π/2,π/2). The effect of the arcsine
transformation is similar to the logit. It pulls out the ends of the
distribution except the extent that the logit does.

The power to detect the hypothetical parameters are 0.65 and 0.45 is
0.48 while the power to detect the difference between hypothetical
parameters 0.25 and 0.05 is 0.82, even though the difference between
both pairs of values is 0.2. The reason is that 0.25 and 0.05 are at the
end which is extreme value of the distribution. Therefore, it takes more
power to detect and arcsine transformation can solve this problem

\subsection{Estimater}\label{estimater}

\section{Case1 MLE of Exponential
Distribution}\label{case1-mle-of-exponential-distribution}

\[f(x;\lambda)=\lambda e^{-\lambda x}\]

\[L(\lambda;X_1,...,X_n)=\lambda e^{-\lambda X_1}\lambda e^{-\lambda X_2}...\lambda e^{-\lambda X_n}\]

\[L(\lambda;X_1,...,X_n)=\lambda^ne^{-\lambda\sum X_i}\]

\[l(\lambda;X_1,...,X_n)=nlog(\lambda)-\lambda \sum X_i\]

\[\frac{dl(\lambda;X1,...,Xn)}{d\lambda}=\frac{n}{\lambda}-\sum X=0\]

\[\hat\lambda=\frac{n}{\sum X_i}=\frac{1}{\bar X_n}\]

\section{\texorpdfstring{Case2 Moment Estimator and MLE for new
distribution
\(\theta\)}{Case2 Moment Estimator and MLE for new distribution \textbackslash{}theta}}\label{case2-moment-estimator-and-mle-for-new-distribution-theta}

MOM:

\begin{align}
E[X]\ & =\ \int^1_0 x((1-\theta)\ +\ 2\theta x)dx \notag \\ 
& =\ (1-\theta)\int^1_0 xdx +\ \int^1_0 2\theta x^2 dx \notag \\ 
& =\ (1-\theta)\frac{1}{2}x^2\arrowvert^1_0\ +\ 2\theta \frac{1}{3}x^3\arrowvert^1_0\ \notag \\ 
& =\ \frac{1}{2} -\ \frac{1}{2}\theta\ +\ \frac{2}{3}\theta \notag \\
& =\ \frac{1}{6}\theta+\frac{1}{2} \notag
\end{align}

MLE:
\[L(\theta;X_1,...,X_n)=[(1-\theta)+2\theta X_1]...[(1-\theta)+2\theta X_n]\]

\[l(\theta;X_1,...,X_n)=log[(1-\theta)+2\theta X_1]+...+log[(1-\theta)+2\theta X_1]\]
In this situation, we cannot taking the derivative of
\(l(\theta;X_1,...,X_n)\) to find the maximum value with corresponding
\(\theta\).

\subsection{Rain in Southern Illinois}\label{rain-in-southern-illinois}

\begin{Shaded}
\begin{Highlighting}[]
\NormalTok{ill.}\DecValTok{60}\NormalTok{ <-}\StringTok{ }\KeywordTok{read.table}\NormalTok{(}\StringTok{"D:/ma677/677final/ill-60.txt"}\NormalTok{, }\DataTypeTok{quote=}\StringTok{"}\CharTok{\textbackslash{}"}\StringTok{"}\NormalTok{, }\DataTypeTok{comment.char=}\StringTok{""}\NormalTok{)}
\NormalTok{yr60<-}\KeywordTok{as.numeric}\NormalTok{(}\KeywordTok{as.array}\NormalTok{(ill.}\DecValTok{60}\NormalTok{[,}\DecValTok{1}\NormalTok{]))}
\NormalTok{ill.}\DecValTok{61}\NormalTok{ <-}\StringTok{ }\KeywordTok{read.table}\NormalTok{(}\StringTok{"D:/ma677/677final/ill-61.txt"}\NormalTok{, }\DataTypeTok{quote=}\StringTok{"}\CharTok{\textbackslash{}"}\StringTok{"}\NormalTok{, }\DataTypeTok{comment.char=}\StringTok{""}\NormalTok{)}
\NormalTok{yr61<-}\KeywordTok{as.numeric}\NormalTok{(}\KeywordTok{as.array}\NormalTok{(ill.}\DecValTok{61}\NormalTok{[,}\DecValTok{1}\NormalTok{]))}
\NormalTok{ill.}\DecValTok{62}\NormalTok{ <-}\StringTok{ }\KeywordTok{read.table}\NormalTok{(}\StringTok{"D:/ma677/677final/ill-62.txt"}\NormalTok{, }\DataTypeTok{quote=}\StringTok{"}\CharTok{\textbackslash{}"}\StringTok{"}\NormalTok{, }\DataTypeTok{comment.char=}\StringTok{""}\NormalTok{)}
\NormalTok{yr62<-}\KeywordTok{as.numeric}\NormalTok{(}\KeywordTok{as.array}\NormalTok{(ill.}\DecValTok{62}\NormalTok{[,}\DecValTok{1}\NormalTok{]))}
\NormalTok{ill.}\DecValTok{63}\NormalTok{ <-}\StringTok{ }\KeywordTok{read.table}\NormalTok{(}\StringTok{"D:/ma677/677final/ill-63.txt"}\NormalTok{, }\DataTypeTok{quote=}\StringTok{"}\CharTok{\textbackslash{}"}\StringTok{"}\NormalTok{, }\DataTypeTok{comment.char=}\StringTok{""}\NormalTok{)}
\NormalTok{yr63<-}\KeywordTok{as.numeric}\NormalTok{(}\KeywordTok{as.array}\NormalTok{(ill.}\DecValTok{63}\NormalTok{[,}\DecValTok{1}\NormalTok{]))}
\NormalTok{ill.}\DecValTok{64}\NormalTok{ <-}\StringTok{ }\KeywordTok{read.table}\NormalTok{(}\StringTok{"D:/ma677/677final/ill-64.txt"}\NormalTok{, }\DataTypeTok{quote=}\StringTok{"}\CharTok{\textbackslash{}"}\StringTok{"}\NormalTok{, }\DataTypeTok{comment.char=}\StringTok{""}\NormalTok{)}
\NormalTok{yr64<-}\KeywordTok{as.numeric}\NormalTok{(}\KeywordTok{as.array}\NormalTok{(ill.}\DecValTok{64}\NormalTok{[,}\DecValTok{1}\NormalTok{]))}
\end{Highlighting}
\end{Shaded}

\begin{Shaded}
\begin{Highlighting}[]
\KeywordTok{library}\NormalTok{(fitdistrplus)}
\end{Highlighting}
\end{Shaded}

\begin{verbatim}
## Warning: package 'fitdistrplus' was built under R version 3.5.3
\end{verbatim}

\begin{verbatim}
## Loading required package: MASS
\end{verbatim}

\begin{verbatim}
## 
## Attaching package: 'MASS'
\end{verbatim}

\begin{verbatim}
## The following object is masked from 'package:dplyr':
## 
##     select
\end{verbatim}

\begin{verbatim}
## Loading required package: survival
\end{verbatim}

\begin{verbatim}
## Loading required package: npsurv
\end{verbatim}

\begin{verbatim}
## Warning: package 'npsurv' was built under R version 3.5.2
\end{verbatim}

\begin{verbatim}
## Loading required package: lsei
\end{verbatim}

\begin{verbatim}
## Warning: package 'lsei' was built under R version 3.5.2
\end{verbatim}

\begin{Shaded}
\begin{Highlighting}[]
\KeywordTok{plotdist}\NormalTok{(yr60)}
\end{Highlighting}
\end{Shaded}

\includegraphics{final_files/figure-latex/use fitdistplus-1.pdf}

\begin{Shaded}
\begin{Highlighting}[]
\KeywordTok{plotdist}\NormalTok{(yr61)}
\end{Highlighting}
\end{Shaded}

\includegraphics{final_files/figure-latex/use fitdistplus-2.pdf}

\begin{Shaded}
\begin{Highlighting}[]
\KeywordTok{plotdist}\NormalTok{(yr62)}
\end{Highlighting}
\end{Shaded}

\includegraphics{final_files/figure-latex/use fitdistplus-3.pdf}

\begin{Shaded}
\begin{Highlighting}[]
\KeywordTok{plotdist}\NormalTok{(yr63)}
\end{Highlighting}
\end{Shaded}

\includegraphics{final_files/figure-latex/use fitdistplus-4.pdf}

\begin{Shaded}
\begin{Highlighting}[]
\KeywordTok{plotdist}\NormalTok{(yr64)}
\end{Highlighting}
\end{Shaded}

\includegraphics{final_files/figure-latex/use fitdistplus-5.pdf}

From the distribution we can get that: the total rainfall for 1960 is
10.574.The total rainfall for 1961 is 13.197.The total rainfall for 1962
is 10.346. The total rainfall for 1963 is 9.71.The total rainfall for
1960 is 7.11.

From the total rainfall of each year, the rainfall in 1961 is definitely
more than the others because its 13.2 units of rainfall in total. From
the distribution of rainfall in year 1960, we can see that this year is
wet because of storms. Even though it is not a lot, it produces much
more rain each time comparing to rainfall in other 4 years. In terms of
all distribution in 5 years, they look quite similar due to the most
rainfall which is concentrated on the left side of the distribution.

\begin{Shaded}
\begin{Highlighting}[]
\NormalTok{all_rainfall<-}\KeywordTok{c}\NormalTok{(yr60,yr61,yr62,yr63,yr64)}


\NormalTok{gamma <-}\StringTok{ }\KeywordTok{fitdist}\NormalTok{(all_rainfall, }\StringTok{"gamma"}\NormalTok{)}
\KeywordTok{plot}\NormalTok{(gamma)}
\end{Highlighting}
\end{Shaded}

\includegraphics{final_files/figure-latex/unnamed-chunk-2-1.pdf}

\begin{Shaded}
\begin{Highlighting}[]
\KeywordTok{summary}\NormalTok{(gamma)}
\end{Highlighting}
\end{Shaded}

\begin{verbatim}
## Fitting of the distribution ' gamma ' by maximum likelihood 
## Parameters : 
##        estimate Std. Error
## shape 0.4408386  0.0337663
## rate  1.9648409  0.2474440
## Loglikelihood:  185.3477   AIC:  -366.6954   BIC:  -359.8455 
## Correlation matrix:
##           shape      rate
## shape 1.0000000 0.6082109
## rate  0.6082109 1.0000000
\end{verbatim}

We fit the gamma distribution into all observed rainfall data of year
between 1960 and 1964. From the density and QQplots, the gamma
distribution fits really well on this data. The authors are right to use
gamma distribution.

\begin{Shaded}
\begin{Highlighting}[]
\NormalTok{f <-}\StringTok{ }\KeywordTok{fitdist}\NormalTok{(all_rainfall, }\StringTok{"gamma"}\NormalTok{,}\DataTypeTok{method =} \StringTok{"mme"}\NormalTok{)}
\NormalTok{b <-}\StringTok{ }\KeywordTok{bootdist}\NormalTok{(f)}
\KeywordTok{summary}\NormalTok{(b)}
\end{Highlighting}
\end{Shaded}

\begin{verbatim}
## Parametric bootstrap medians and 95% percentile CI 
##          Median      2.5%     97.5%
## shape 0.3882525 0.2730412 0.5260771
## rate  1.7327806 1.1453228 2.5510982
\end{verbatim}

\begin{Shaded}
\begin{Highlighting}[]
\NormalTok{fgamma <-}\StringTok{ }\KeywordTok{fitdist}\NormalTok{(all_rainfall, }\StringTok{"gamma"}\NormalTok{,}\DataTypeTok{method =} \StringTok{"mle"}\NormalTok{)}
\NormalTok{bgmm <-}\StringTok{ }\KeywordTok{bootdist}\NormalTok{(fgamma)}
\KeywordTok{summary}\NormalTok{(bgmm)}
\end{Highlighting}
\end{Shaded}

\begin{verbatim}
## Parametric bootstrap medians and 95% percentile CI 
##          Median      2.5%     97.5%
## shape 0.4418645 0.3811267 0.5131649
## rate  1.9764049 1.5522837 2.5186370
\end{verbatim}

For method of moment, the 95\% confidence interval of shape from
bootstrap sample is (0.2761133,0.5290801), the rate is
(1.1749219,2.5559406). For the MLE, the 95\% confidence interval of
shape from bootstrap sample is (0.3852016,0.514886),the rate is
(1.5715661,2.599747). Obviously, the MLE estimates have wider CI and
lower variances.The MLE is better due to its lower variance

\subsection{Decision Theory}\label{decision-theory}


\end{document}
